\begin{tcolorbox}[sharp corners, colback=white,boxrule=1mm]
	\section{Definição}
	\subsection{Dependência funcional}
	Representação: $\alpha \to \beta$
	
	Condição: Para todo $t_1$ e $t_2$:
	
	\[t_1[\alpha] = t_2[\alpha] \to t_1[\beta] = t_1[\beta]\]
	
	\subsubsection{Dependência funcional trivial}
	
	A dependência funcional $\alpha \to \beta$ é trivial se $\beta \subseteq \alpha$
	
	\subsection{Dependência de valores múltiplos}
	Representação: $\alpha \to\to \beta$
	
	Condição: Para todos os pares $t_1$ e $t_2$, tais que $t_1[\alpha] = t_2[\alpha]$ existem tuplas $t_3$ e $t_4$ em $r$ tais que:
	
	\begin{gather*}
		t_1[\alpha] = t_2[\alpha] = t_3[\alpha] = t_4[\alpha]\\
		t_3[\beta] = t_1[\beta]\\
		t_3[R - \beta] = t_2[R-\beta]\\
		t_4[\beta] = t_2[\beta]\\
		t_4[R - \beta] = t_1[R-\beta]
	\end{gather*}

	\subsection{Parcialmente dependente}
	$\alpha \to \beta$ é parcial se:
	\[\exists \gamma: \gamma \subset \alpha \wedge \gamma \to \beta\]
	
	Dessa forma, $\beta$ é parcialmente dependente de $\alpha$
	
	\subsection{Transitivamente dependente}
	Um atributo $A$ é transitivamente dependente de $\alpha$ se existe um conjunto de atributos $\beta$ tais que:
	\begin{gather*}
		\alpha \to \beta\\
		\neg(\beta \to \alpha)\\
		A \not\in \alpha\\
		A \not\in \beta\\
		\beta \to A
	\end{gather*}
\end{tcolorbox}

\newpage
\begin{tcolorbox}[sharp corners, colback=white,boxrule=1mm]
	\section{Formas normais}
	\subsection{Primeira forma}
	Definições alternativas:
	\begin{itemize}
		\item Nenhum atributo possui estrutura
		\item Todos os atributos são atômicos
	\end{itemize}
	
	\subsection{Segunda forma}
	Para cada atributo $A$, ele é primo, ou não está parcialmente dependente de uma chave candidata
	
	\subsection{Terceira forma}
	\subsubsection{Através de dependências funcionais}
	Para cada esquema de relação $R$, para cada dependência funcional $\alpha \to \beta$ em $F^+$, pelo menos uma dessas sentenças é verdadeira:
	
	\begin{itemize}
		\item $\alpha \to \beta$ é trivial 
		\item $\alpha \to R$ ($\alpha$ é superchave)
		\item Cada atributo $\beta - \alpha$ é primo
	\end{itemize}
	
	\subsubsection{Através de dependências transitiva}
	Nenhum atributo não primo é transitivamente dependente.
	
	\subsection{Forma normal Boyce-Codd}
	Para todas as dependência funcional em $F+$, pelo menos um desses é verdadeiro:
	
	\begin{itemize}
		\item é trivial
		\item $\alpha \to R$
	\end{itemize}
	obs.: decomposição em FNBC não preserva dependência funcional
	
	\subsection{Quarta forma normal}
	Para toda dependência $\alpha\to\to\beta$, um deles se aplica:
	\begin{itemize}
		\item é trivial
		\item $\alpha \to R$
	\end{itemize} 
\end{tcolorbox}