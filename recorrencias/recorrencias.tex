\begin{tcolorbox}[sharp corners, colback=white,boxrule=1mm]
\section{Formulas para resoluções de recorrências}
\subsection{Linear de primeira ordem}
De:
\[S(n) = cS(n-1) + g(n)\]
Para:
\[S(n) = c^{n-1}S(1) + \sum_{i=2}^{n}c^{n-i}g(i)\]
Em alguns contextos da computação:
\[S(n) = c^{n-1}S(0) + \sum_{i=1}^{n}c^{n-i}g(i)\]

\subsection{Linear de segunda ordem}
De:
\[S(n) = c_1S(n-1) + c_2S(n-2)\]
Resolver a equação característica:
\[t^2 - c_1t - c^2 = 0\]

Se tiver 2 raízes distintas $r_1$ e $r_2$:
\[S(n) = pr_1^{n-1} + qr_2^{n-1}\]

em que:
\[p + q = S(1)\]
\[pr_1 + qr_2 = S(2)\]

Se tiver 2 raízes iguais \(r\):
\[S(n) = pr^{n-1} + q(n-1)r^{n-1}\]

em que:
\[p = S(1)\]
\[pr + qr = S(2)\]

\subsection{recorrência dividir para conquistar}
de:
\[S(n) = c\cdot S\left ( \frac n2 \right ) + g(n)\]

Para:
\[S(n) = c^{\lg n}S(1) + \sum_{i=1}^{\lg n}c^{(\lg n) - i}g(2^i)\]

\subsection{Teorema mestre}
\[S(1) \geq 0\]
\[S(n) = aS\left(\frac{n}{2}\right) + n^c\]
Para \(n \geq 2\) e \(n = b^m\)

\[a < b^c \rightarrow S(n) = \Theta(n^c)\]
\[a = b^c \Rightarrow S(n) = \Theta(n^c \log n)\]
\[a > b^c \Rightarrow S(n) = \Theta(n^{\log_b a})\]
\end{tcolorbox}